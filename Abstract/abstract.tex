% ************************** Thesis Abstract *****************************
% Use `abstract' as an option in the document class to print only the titlepage and the abstract.
\begin{abstract}
The Boolean Satisfiability Problem (SAT) is a well studied NP-COMPLETE
problem for which there exists no known algorithm faster than 
$\mathcal{O}^{\ast}(2^n)$
where $n$ is the number of variables. However, fast ``in practice'' SAT solvers
have been developed that are used in a wide range of domains. We give an
introduction into how these solvers function and contrast their performance with the
conditional lower bounds that can be shown via hypotheses about the complexity
of SAT. We detail the sparsification lemma, a critical technique used to show these
lower bounds. Finally, we cover observations about the structure of SAT that
modern solvers exploit to achieve speeds faster than what worst case complexity
would suggest.

\emph{Keywords}: SAT, SAT Solvers, Conditional Lower Bounds, Subexponential Time, Phase Transition, Backdoor sets, Sparsification Lemma
\end{abstract}
